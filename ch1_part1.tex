\documentclass[english,10pt,table]{beamer}
%\documentclass[english,10pt,handout]{beamer}
%\documentclass[slidestop,usepdftitle=false]{beamer}

\input{aa-style.tex}
\lecture[1]{Introduction Python}{lecture-text}

%\date[]{~~}
%%%%%%%%%%%%%%%%%%%%%%%%%%%%%%%%%%%%%%%%%%%%%%%%%%%%%%%%%%%%%%%%%%%%%
%%%%%%%%%%%%%%%%%%%%%%%%%%%%%%%%%%%%%%%%%%%%%%%%%%%%%%%%%%%%%%%%%%%%%
\begin{document}

\begin{frame}
  \maketitle
\end{frame}

\section*{Contents}
\begin{frame}\frametitle<presentation>{Contents}
  \tableofcontents
\end{frame}


%%%%%%%%%%%%%%%%%%%%%%%%%%%
\section{Definition and notations}
\subsection{Definition}
\begin{frame}{Definition}		
	\begin{block}{What is a python?}
	Python is a cross-platform programming language.
  \end{block}	
\pause
		
	\begin{block}{Properties of python}
	 \begin{itemize}
		 \item \alert{Design by} Guido van Rossum (1991),
		 \item \alert{Paradigm} multi-paradigm, object-oriented, imperative, functional, procedural, reflective,
		 \item \alert{Typing discipline} duck, dynamic, strong,
   \end{itemize}
  \end{block}	
\end{frame}

%%%%%%%%%%%%%%%%%%%%%%%%%%%%
\subsection{Complexity}
\begin{frame}{Complexity}
		
	\begin{block}{}
	 	 \begin{itemize}\small
		 \item Generally, not much interested in time and space complexity for small inputs.
		 \item Given two algorithms \alert{$A$} and \alert{$B$} for solving problem $P$.
		\end{itemize}
  \end{block}	
	
	\begin{center}
	\small\begin{tabular}{|c|c|c|}
			\hline
			Input size & Algorithm A & Algorithm B \\ \hline
		$n$ & \textcolor{blue}{5000 $n$} & \textcolor{red}{$1.2^n$}\\ \hline
		$10$ & \textcolor{blue}{50,000} & \textcolor{red}{6}\\ \hline
		$100$ & \textcolor{blue}{500,000} & \textcolor{red}{2,817,975}\\ \hline
		$1,000$ & \textcolor{blue}{5,000,000} & \textcolor{red}{$1.5 \times 10^{79}$}\\ \hline
		$100,000$ & \textcolor{blue}{$5 \times 10^8$} & \textcolor{red}{$1.3 \times 10^{7918}$}\\ \hline
	\end{tabular}
 \end{center}
	
	\begin{block}{}
	 	 \begin{itemize}\small
		 \item <2-> \alert{$B$} cannot be used for large inputs, while \alert{$A$} is still feasible.
		 \item <2-> So what is important is the \alert{growth} of the complexity functions.
		 \item <2-> Growth of time and space complexity with increasing input size \alert{$n$} is a suitable measure for the comparison of algorithms. 
		\end{itemize}
  \end{block}		
\end{frame}

%%%%%%%%%%%%%%%%%%%%%%%%%%%%
\subsection{Formulas}
\begin{frame}{Types of formulas for basic operation count}
		
	\begin{block}{}
	 	 \begin{itemize}\small
		 \item Exact formulas, e.g., \alert{$C(n) = n(n-1)/2$}.
		 \item <2-> Formula indicating order of growth with specific multiplicative constant
            e.g., \alert{$C(n) \approx 0.5 n^2$}.
		 \item <3-> Formula indicating order of growth with unknown multiplicative constant
            e.g., \alert{$C(n) \approx c.n^2$}
		 \item <4-> Most important: Order of growth within a constant multiple as \alert{$n \rightarrow \infty$}
		\end{itemize}
  \end{block}	
 \only<5>{	
	\begin{block}{Asymptotic growth rate}
	 	 A way of comparing functions that ignores constant factors and small input sizes
		\begin{itemize}\small
		 \item \alert{$O(g(n))$}: class of functions \alert{$f(n)$} that grow \alert{no faster than} \alert{$g(n)$}
 		 \item \alert{$\Theta(g(n))$}: class of functions \alert{$f(n)$} that grow \alert{at the same rate as} \alert{$g(n)$}
		 \item \alert{$\Omega(g(n))$}: class of functions \alert{$f(n)$} that grow \alert{at least as fast as} \alert{$g(n)$}
		\end{itemize}
  \end{block}	
 }
\end{frame}


%%%%%%%%%%%%%%%%%%%%%%%%%%%%
\begin{frame}{Complexity classes - a small vocabulary}
	\begin{block}{}
	 \begin{itemize}\small
 		  \item \textcolor{blue}{Constant}: \alert{$O(1)$} (independing on the input size)
     \item \textcolor{blue}{Sub-linear} or \textcolor{blue}{logarithmic}: \alert{$O(\log n)$}
     \item \textcolor{blue}{Linear}: \alert{$O(n)$}
					\item \textcolor{blue}{Quasi-linear}: \alert{$O(n \log n)$}
					\item \textcolor{blue}{Quadratic}: \alert{$O(n^2)$}
					\item \textcolor{blue}{Cubic}: \alert{$O(n^3)$}
     \item \textcolor{blue}{Polynomial}: \alert{$O(n^p)$} (\alert{$O(n^2)$}, \alert{$O(n^3)$}, etc)
     \item \textcolor{blue}{Quasi-polynomial}: \alert{$O(n^{\log(n)})$}
     \item \textcolor{blue}{Exponential}: \alert{$O(2^n)$}
					\item \textcolor{blue}{Factorial}: \alert{$O(n!)$}
		\end{itemize}
 \end{block}	
\end{frame}


\begin{frame}{Asymptotic upper bound - worst case}
	
	\begin{block}{Asymptotic upper bound  ``big O''}\small
		\textcolor{blue}{$T(n)$ = \alert{$O$}$(f(n))$} \textbf{iif} 
		\alert{$\exists c \in R^+$}, \alert{$c>0$} and \alert{$\exists n_0 \in N$}, \alert{$n_0>0$} \\
		such that \alert{$\forall n>n_0$}: \alert{$T(n) \leq c \times f(n) $}
 \end{block}	
	\pause
		\begin{block}{Example}\small
		Let \alert{$T(n) = 2n+3n^3 + 5$}. \alert{$T(n)$} is in \alert{$O(n^3)$} with:
	 \begin{itemize}
     \item (\alert{$c=8$} and \alert{$n_0=1$}) or (\alert{$c=5$} and \alert{$n_0=2$})
		\end{itemize}
 \end{block}	
	\pause
	\begin{block}{}\small
		\textbf{Principle}: \textcolor{blue}{the lower-order terms are negligible.}
 \end{block}		
\end{frame}


\begin{frame}{Asymtotic lower bound - best case}
	
	\begin{block}{``big Omega''}\small
		\textcolor{blue}{$T(n)$ = \alert{$\Omega$}$(f(n))$} \textbf{iif} 
		\alert{$\exists c \in R^+$}, \alert{$c>0$} and \alert{$\exists n_0 \in N$}, \alert{$n_0>0$}\\
		such that \alert{$\forall n>n_0$}: \alert{$T(n) \geq c \times f(n) $}
 \end{block}	
	\pause
		\begin{block}{Example}\small
		Let \alert{$T(n) = 2n+3n^3 + 5$}. \alert{$T(n)$} is in \alert{$\Omega(n^3)$} with:
	 \begin{itemize}
     \item (\alert{$c=1$} and \alert{$n_0=1$}) 
		\end{itemize}
 \end{block}	
\end{frame}


\begin{frame}{Asymtotic approximating bound - average case}
	
	\begin{block}{``big Theta''}
  \textcolor{blue}{$T(n)$ = \alert{$\Theta$}$(f(n))$} \textbf{iif}
		\alert{$\exists c_1, c_2 \in R^+$}, \alert{$c_1>0$}, \alert{$c_2>0$} and \alert{$\exists n_0 \in N$}, \alert{$n_0>0$}\\
		such that \alert{$\forall n>n_0$}: \alert{$c_1 \times f(n) \leq T(n) \leq c_2 \times f(n) $}
 \end{block}	
		\pause
		\begin{block}{Property}\small
		\alert{$T(n) = O(f(n))$} and \alert{$T(n) = \Omega(f(n))$} $\Longrightarrow$ \alert{$T(n) = \Theta(f(n))$}
 \end{block}	
	\pause
	\begin{block}{Example}\small
		Let \alert{$T(n) = 2n+3n^3 + 5$}. \\
		So, \alert{$T(n)$} is in \alert{$O(n^3)$} and in \alert{$\Omega(n^3)$}.\\
		Consequently, \alert{$T(n)$} is in \alert{$\Theta(n^3)$}.
 \end{block}	
\end{frame}

%%%%%%%%%%%%%%%%%%%%%%%%%%%%
\begin{frame}{Other properties}
	
	\begin{block}{Not transitive}\footnotesize
	 \begin{itemize}
     \item $f(n) = n^2$; $g(n)= n$
		 \item $\Rightarrow f(n) = O(n^2) = g(n)$ but $f(n) \neq g(n)$
	\end{itemize}
 \end{block}
\pause
   \begin{block}{Transitivity}\small				
			\begin{itemize}\small				
		   \item $f(n) = O(g(n))$ \& $g(n) = O(h(n))$ $\Rightarrow f(n) = O(h(n))$
			 \item $f(n) = \Omega(g(n))$ \& $g(n) = \Omega(h(n))$ $\Rightarrow f(n) = \Omega(h(n))$
			 \item $f(n) = \Theta(g(n))$ \& $g(n) = \Theta(h(n))$ $\Rightarrow f(n) = \Theta(h(n))$
      \end{itemize}
   \end{block}
\pause
    \begin{block}{Additivity}\small				
			\begin{itemize}\small				
		   \item $f(n) = O(h(n))$ \& $g(n) = O(h(n))$ $\Rightarrow f(n) + g(n) = O(h(n))$
			 \item $f(n) = \Omega(h(n))$ \& $g(n) = \Omega(h(n))$ $\Rightarrow f(n) + g(n) = \Omega(h(n))$
			 \item $f(n) = \Theta(h(n))$ \& $g(n) = \Theta(h(n))$ $\Rightarrow f(n) + g(n) = \Theta(h(n))$
      \end{itemize}
   \end{block}
\end{frame}

%%%%%%%%%%%%%%%%%%%%%%%%%%%%
\begin{frame}{Exercise}
	
	\begin{block}{}\footnotesize
	 \begin{itemize}
     \item <2->\alert{$T(n) = 3 + 5n^2$} $\Rightarrow$ \alert{$T(n) = \Theta(n^2)$} ?
		 \item <3->if \alert{$T(n) = 
		    \left\{ \begin{array}{cccc} 
			    2n + 5 &  if & n & \mbox{ is even }\\
				  n^2 - n + 1 &  if & n & \mbox{ is odd }\\
								\end{array} \right.$}, then \alert{$T(n) = O(?)$} and \alert{$T(n) = \Omega(?)$}.
		\end{itemize}
 \end{block}	
\end{frame}
%%%%%%%%%%%%%%%%%%%%%%%%%%%%
\begin{frame}{Exercise}
	
	\begin{block}{}
  Compare the asymptotic behaviours of 
	\begin{enumerate}
		\item \alert{$2^n$} and \alert{$10^n$}
		\item \alert{$\log_{2}n$} and \alert{$\log_{10}n$}
	\end{enumerate}	
 \end{block}

\pause

	\begin{block}{}
	\begin{enumerate}
  \item Prove that for any positive functions \alert{$f$} and \alert{$g$}, \alert{$f(n) +g(n)$} and \alert{$max(f(n); g(n))$} are asymptotically equivalent. \\
	\item Give a (necessary and sufficient) condition on positive functions \alert{$f$} and \alert{$g$} to ensure that \alert{$f(n) +g(n)$} and \alert{$f(n)$} are asymptotically equivalent. 
	\end{enumerate}
	\end{block}	
\end{frame}

%%%%%%%%%%%%%%%%%%%%%%%%%%%%

\begin{frame}{Common asymptotic behaviours}
	\begin{block}{}\footnotesize
	\begin{tabular}{|c|c|c|c|c|c|c|}
		\hline
		Size & \multicolumn{6}{c|}{Approximate computational time}\\ \hline
		$n$ & $\Theta(\log n)$ & $\Theta(n)$ & $\Theta(n \log n)$ & $\Theta(n^2)$ & $\Theta(2^n)$ & $\Theta(n!)$ \\ \hline
		$10$ & $3. 10^{-9}$s & $10^{-8}$s & $3. 10^{-8}$s & $10^{-7}$s & $10^{-6}$s & $3. 10^{-3}$s\\
		$10^2$ & $7. 10^{-9}$s & $10^{-7}$s & $7. 10^{-7}$s & $10^{-5}$s & $4. 10^{13}$y  & * \\
		$10^3$ & $10^{-8}$s & $10^{-6}$s & $10^{-5}$s & $10^{-3}$s  & * & * \\
		$10^4$ & $1,3. 10^{-8}$s & $10^{-5}$s & $10^{-4}$s & $10^{-1}$s & * & * \\
		$10^5$ & $1,7. 10^{-8}$s & $10^{-4}$s & $2. 10^{-3}$s & $10$s & * & * \\
		$10^6$ & $2. 10^{-8}$s & $10^{-3}$s & $2. 10^{-2}$s & $17$m & * & * \\ \hline
	\end{tabular}
  \end{block}
\end{frame}



%%%%%%%%%%%%%%%%%%%%%%%%%%%%
\section{Basic methods for asymptotic behaviour analysis}
\subsection{Counting number of elementary operations}
\begin{frame}{Example}
		
	\begin{block}{}
	 \begin{enumerate}[1.]\small
		   \item \only<2->{\textcolor{orange}{(1)}}\textbf{Var} int: \alert{$d=0$} 
	    \item \only<3->{\textcolor{orange}{($n$)}}\textbf{For} \alert{$i$} from \alert{$1$} to \alert{$n$} \textbf{do}
				 	\begin{enumerate}[1.]\small
	      \item \only<4->{\textcolor{orange}{(1)}}\alert{$d$} = \alert{$d + 1$}
	      \item \only<5->{\textcolor{orange}{(1)}}\alert{$a[i]$} = \alert{$a[i] \times a[i] + d \times d$}
						\end{enumerate}
					\item \textbf{Endfor}
    \end{enumerate}
  \end{block}	
		\only<6->{
		\begin{block}{}
		 Number of elementary operations: \alert{$1 + n \times ( 1 + 1) = 2n +1$}.
	 \end{block}	
		}
\end{frame}
%%%%%%%%%%%%%%%%%%%%%%%%%%%
\begin{frame}{Linear loop example}
  \begin{columns}
  \begin{column}{0.48\textwidth}
   \begin{block}{}\small				
			\begin{enumerate}[1.]\small				
		   \item \textbf{Var} int: \alert{$i=1$}
     \item \textbf{While} \alert{$i \leq n$} \textbf{do}
					 \begin{enumerate}[1.]\small				
       \item Write ``Bonjour'' 
							\item \alert{$i = i + 1$}
						\end{enumerate}
					\item \textbf{EndWhile}
			\end{enumerate}
   \end{block}
	  \end{column}
			\pause
   \begin{column}{0.48\textwidth}
    \begin{block}{}\small				
			\begin{enumerate}[1.]\small				
		   \item \textbf{Var} int: \alert{$i=n$}
     \item \textbf{While} \alert{$i \geq 1$} \textbf{do}
					 \begin{enumerate}[1.]\small				
       \item Write ``Bonjour'' 
							\item \alert{$i = i - 1$}
						\end{enumerate}
					\item \textbf{EndWhile}
			\end{enumerate}
   \end{block}
  \end{column}
  \end{columns}
	
		\only<3->{
		\begin{block}{}
		 Number of elementary operations: \alert{$2n+1$}.
	 \end{block}	
		}
\end{frame}

%%%%%%%%%%%%%%%%%%%%%%%%%%%
\begin{frame}{Logarithmic loop example}
  \begin{columns}
  \begin{column}{0.48\textwidth}
   \begin{block}{}\small				
			\begin{enumerate}[1.]\small				
		   \item \textbf{Var} int: \alert{$i=1$}
     \item \textbf{While} \alert{$i \leq n$} \textbf{do}
					 \begin{enumerate}[1.]\small				
       \item Write ``Bonjour'' 
							\item \alert{$i = i \times 2$}
						\end{enumerate}
					\item \textbf{EndWhile}
			\end{enumerate}
   \end{block}
	  \end{column}
			\pause
   \begin{column}{0.48\textwidth}
    \begin{block}{}\small				
			\begin{enumerate}[1.]\small				
		   \item \textbf{Var} int: \alert{$i=n$}
     \item \textbf{While} \alert{$i \geq 1$} \textbf{do}
					 \begin{enumerate}[1.]\small				
       \item Write ``Bonjour'' 
							\item \alert{$i = i / 2$}
						\end{enumerate}
					\item \textbf{EndWhile}
			\end{enumerate}
   \end{block}
  \end{column}
  \end{columns}
		
		\only<3->{
		\begin{block}{}
		 Number of elementary operations: \alert{$1 + \log_2 (n)$}.
	 \end{block}	
		}
\end{frame}

%%%%%%%%%%%%%%%%%%%%%%%%%%%
\begin{frame}{Nested loop example}
   \only<1->{
			\begin{block}{}\small			
     \textcolor{blue}{Nb of iterations = nb of iterations of external loop $\times$ nb of iterations of internal loop}
			\end{block}
			}
			\pause
			\begin{columns}
  \begin{column}{0.48\textwidth}			
    \begin{block}{}\small				
			\begin{enumerate}[1.]\small				
		   \item \textbf{Var} int: \alert{$i=1$}
     \item \textbf{While} \alert{$i \leq n$} \textbf{do}
					 \begin{enumerate}[1.]\small				
       \item \textbf{Var} int: \alert{$j=1$}
							\item \textbf{While} \alert{$j \leq n$} \textbf{do}
					   \begin{enumerate}[1.]\small				
         \item Write ``Bonjour'' 
									\item \alert{$j = j \times 3$}
							 \end{enumerate}
					  \item \textbf{EndWhile}
							\item \alert{$i = i + 1$}
						\end{enumerate}
					\item \textbf{EndWhile}
			\end{enumerate}
   \end{block}
		 \end{column}
			\pause
   \begin{column}{0.48\textwidth}
		\begin{block}{}
		 Number of elementary operations: \alert{$1+ n + n \times \log_3 (n)$}.
	 \end{block}	
  \end{column}
  \end{columns}
\end{frame}

%%%%%%%%%%%%%%%%%%%%%%%%%%%
\begin{frame}{Exercise}
	\begin{block}{Function XYZ(array: $a[]$)}\small				
			\begin{enumerate}[1.]\small				
		   \item \only<2->{\textcolor{orange}{(1)}}\textbf{Var} int: \alert{$i$}
     \item \only<3->{\textcolor{orange}{($n$)}}\textbf{For} \alert{$i$} from \alert{$1$} to \alert{$n$} \textbf{do}
					 \begin{enumerate}[1.]\small				
       \item \only<4->{\textcolor{orange}{(1)}}\textbf{Var} int: \alert{$t$} = \alert{$a[i]$}
							\item \only<5->{\textcolor{orange}{(1)}}\textbf{Var} int: \alert{$j$}
							\item \only<6->{\textcolor{orange}{(?)}}\textbf{For} \alert{$j$} from \alert{$i-1$} to \alert{$0$} \textbf{do}
							 \begin{enumerate}[1.]\small				
         \item \only<7->{\textcolor{orange}{(1)}}\alert{$a[j+1]$} = \alert{$a[j]$}
								\end{enumerate}
					  \item \textbf{EndFor}
							\item \only<8->{\textcolor{orange}{(1)}}\alert{$a[j+1]$} = \alert{$t$}
						\end{enumerate}
					\item \textbf{EndFor}
			\end{enumerate}
 \end{block}		
\end{frame}
%%%%%%%%%%%%%%%%%%%%%%%%%%%%



%%%%%%%%%%%%%%%%%%%%%%%%%%%%
\begin{frame}{Homeworks}
	
	\begin{block}{}\footnotesize
	\textcolor{blue}{Give algorithms having number of elementary operations as below.}\\
	 \begin{itemize}
     \item \alert{$T_1(n) = 3 + 5n$}
					\item \alert{$T_2(n) = n \log_2 n$}
					\item \alert{$T_3(n) = n^3$}
					\item \alert{$T_4(n) = (3n)!$}
					\item \alert{$T_5(n) =  \log_2 (3 n) $}
					\item \alert{$T_6(n) = 2 \log_3 (2 n)$}
					\item \alert{$T_7(n) = n^2 \log_4 n $}
					\item \alert{$T_8(n) = \sqrt{n} $}
					\item \alert{$T_{9}(n) = \sqrt[3]{n^2} $}
					\item \alert{$T_{10}(n) = 2^n $}
					\item \alert{$T_{11}(n) = n! $}
		\end{itemize}
 \end{block}	
\end{frame}
%%%%%%%%%%%%%%%%%%%%%%%%%%%%

\end{document}